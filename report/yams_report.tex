% Report for YAMS, project in CWRU EECS 314 Spring 2015
%
% Members:
%    Stephen Brennan (smb196)
%    Katherine Cass (krc53)
%    Jeffrey Copeland (jpc86)
%    Andrew Mason (ajm188)
%    Thomas Murphy (trm70)
%    Aaron Neyer (agn31)

\documentclass[journal,10pt]{IEEEtran}
% Will correctly load from ./
% Explicit use of 10pt by report requirements

% *** GRAPHICS RELATED PACKAGES ***
%
\ifCLASSINFOpdf
 \usepackage[pdftex]{graphicx}
\else
 \usepackage[dvips]{graphicx}
\fi

% declare the path(s) where your graphic files are
 \graphicspath{{screenshots/}}
% declare graphics file extensions
\DeclareGraphicsExtensions{.pdf,.jpeg,.png}

% *** MATH PACKAGES ***
%
\usepackage[cmex10]{amsmath}

% *** PDF, URL AND HYPERLINK PACKAGES ***
%
\usepackage{url}
% url.sty was written by Donald Arseneau. It provides better support for
% handling and breaking URLs.
% Basically, \url{my_url_here}.

% Correctly create hyperlinks with URLs
\usepackage{hyperref}

% Allow nicer float options
\usepackage{float}

% correct bad hyphenation here
\hyphenation{op-tical net-works semi-conduc-tor}


\begin{document}
% paper title
\title{YAMS: Awesome MIPS Server}

% author names
\author{
Stephen~Brennan,
Katherine~Cass,
Jeffrey~Copeland,
Andrew~Mason,
Thomas~Murphy,
Aaron~Neyer% stop_space
\thanks{All authors are with the Department of Electrical Engineering and Computer Science, Case Western Reserve University, Cleveland, OH, 44106 USA}% stop_space
\thanks{Final Report submitted May 1, 2015, typographical revisions and clarifications \today}
} % Close the author environment


% The paper headers
\markboth{CWRU EECS 314 Spring 2015 Final Project}%
{}
% The only time the second header will appear is for the odd numbered pages
% after the title page when using the twoside option.

% make the title area
\maketitle

% As a general rule, do not put math, special symbols or citations
% in the abstract or keywords.
\begin{abstract}
We set out to build a simple web server in MIPS. Mission Accomplished.
\end{abstract}

% Note that keywords are not normally used for peerreview papers.
\begin{IEEEkeywords}
MIPS, computer architecture, HTTP, server, MARS, ISA simulation.
\end{IEEEkeywords}

% Begin body of the report

\section{Problem Statement}

\IEEEPARstart{T}{he} goal of this project was to write a static HTTP server in
MIPS assembly running in the MARS simulator. The server would be able to serve a
website, which is contained in the \texttt{html/} directory on port 19001, which
can be viewed in a browser. Sockets were used for networking and were made
available to MIPS by extending MARS syscalls.


\section{Major Challenges}

One major challenge in YAMS was implementing socket syscalls.
MARS was chosen early on because it allowed for custom syscalls; however,
limited documentation is available. This required us to learn about MARS
through lots of trial-and-error. Additionally, if one stops MARS while
waiting on a socket syscall, the system enters weird error states, forcing
us to either 1) double-reassemble-run the program, or 2) restart all of MARS.
This made debugging much slower than a normal MIPS program.

The larger challenge in implementing YAMS was parsing HTTP requests. RFC
2616\cite{Leach}, our HTTP (Hypertext Transfer Protocol) reference for this
project, is 175 pages long, While it is possible to implement a full HTTP stack
in assembly, it is difficult to do in a month or two. Thus, we focused on what
was required to get web pages to display in a browser: GET, POST, Expect, and
only identity encoding or Content-Length. Additionally, because the request
parser had to interact with the socket depending on various headers (e.g.
Transfer-Encoding, Content-Length, and Expect) Testing and debugging revolved
around trial-and-error requests with the Unix program \texttt{curl} or a browser.

\section{System Components}

Upon submission of the report, the entirety of the implementation will be available at \cite{Brennan}. The breakdown of the system components is as follows.

\subsection{String Operations}

In order to parse HTTP requests, we needed to implement a fair amount of the C
standard string library.  In \texttt{mips/string.asm}, we implemented
\texttt{strlen}, \texttt{strncpy}, \texttt{memcpy}, \texttt{strcmp},
\texttt{strncmp}, \texttt{atoi}, \texttt{htoi}, \texttt{strcat},
\texttt{strncat}, as well as two functions for identifying the index of
characters and substrings, \texttt{str\_index\_of} and
\texttt{substr\_index\_of} respectively.  To verify their correctness, we wrote
tests in \texttt{mips/test\_string.asm}.  They can be run in MARS by loading
that file as the main file, instead of \texttt{mips/main.asm}.

\subsection{HTTP Request Handling}

As mentioned in Major Challenges, our focus with request parsing was on serving
a page that a browser could render. Therefore, we support a subset of HTTP
request statuses (e.g. GET, POST) look for a few headers (e.g. Content-Length,
Expect) and read or write accordingly. The logic for request parsing is in
\texttt{http-requests.asm}, and is reached through the \texttt{get\_request}
method from main, returning an internal HTTP request type, and, as applicable,
request URI, request body, request body length, and content type. The main
method takes this data, and hands it off to the file loader and response builder.

The request parser was determined to be working through trial-and-error, seeing
if debug prints contained the right information and if the browser loaded the
page properly. The mixture of MIPS assembly and MARS syscall extensions
precluded the use of unit tests to verify the correctness of the parser's
behavior.

\subsection{File Access}

Accessing files was already possible through MARS syscalls, saving us the
trouble of adding this feature to the simulator. These calls were wrapped in
function-like macros for ease of use.

Additionally, the HTTP handling receives URIs (uniform resource identifiers),
not system paths. These strings needed processing to produce a valid file path
that could be consumed by the MARS syscall. Our processing, implemented in
\texttt{mips/file.asm}, is composed of substring-blacklisting to prevent the
\texttt{../} directory from being used to escape into uncontrolled resources and
string operations to modify the URI into a useful file path. The file path is
created using the default \texttt{html/} directory (where our web resources are
stored) as the base relative file path and URIs with a trailing \texttt{/} have
the file name \texttt{index.html} appended. The processing operation has tests
implemented in \texttt{mips/test\_file.asm}.

\subsection{Turing Tape Language Interpreter}

As an additional ``stretch'' feature, we implemented an interpreter for the
esoteric programming language Brainfuck\cite{Mpreu/preller}.  This simple
language simulates a Turing machine's operations and has only 8 commands (each
of which is a single character).  The interpreter is implemented in
\texttt{mips/brainfuck.asm}, and tested in \texttt{mips/test\_brainfuck.asm}.
When the server is running, code can be loaded into the interpreter by sending a
POST request to the URL \texttt{/load}.  The code can then be run by sending
input in a POST request to the URL \texttt{/run}, which will return the program
output in its response.  As a more simple interface, the URL
\texttt{/brainfuck.html} contains a simple page that uses JavaScript to send
code and input, and receive output, from the interpreter.

\section{Component Integration}

In order to integrate these different components into a cohesive project, we had
to define and follow a strict ``calling convention.''  We decided that all
functions would be called using the \texttt{JAL} instruction.  Only the
\texttt{\$s0-\$s7} saved registers, and global pointers, were to be preserved
across function calls. Arguments were passed (and not necessarily preserved) in
the \texttt{\$a0-\$a3} registers, and return values were placed in the
\texttt{\$v0-\$v1} registers. Any additional arguments or return values were to
be placed on the stack. It was the caller's responsibility to push any other
registers which it wants preserved to the stack.  To facilitate these rules, we
defined macros in \texttt{mips/util-macros.asm} for pushing and popping from the
stack, to make that operation semantically clear in our code.

\section{User Interface}

We have implemented an HTTP server, so there isn't really a ``user interface''
per se. To compile MARS with our implemented syscalls, run \texttt{make} from
the project root. Then run \texttt{java -jar Mars4\_5-SockMod.jar}. You are now
running the modified version of MARS. You can then load in the
\texttt{mips/main.asm} file. Assemble and run that, and you will have a server
listening on port 19001 (\url{http://localhost:19001/}). Now, you can point your
browser to any file in the \texttt{html/} directory. Leave off the
\texttt{html/} part of the path - that directory is considered to be root by the
server. Alternatively, you can use the Unix tool \texttt{curl}.

\section{Documentation of Runs}

\subsection{Static Page Content}

The static page in Fig.~\ref{fig:static_document} is hosted as the
\texttt{/index.html} document of the server. This is the document the user would
expect to load into their browser upon access to the root page of the server.
This demonstrates the ability to serve multiple resources for a page: images,
fonts, and the HTML.

\subsection{Dynamic JavaScript Content}

The dynamic page in Fig.~\ref{fig:dynamic_document} is the presentation given to
the class on April 23, 2015. This document uses HTML, JavaScript, and images.
All resources are loaded from YAMS and not from external internet hosts.

\subsection{Interactive AJAX Content}

The final dynamic-content page, shown in Fig.~\ref{fig:interactive_ajax}, is the
front end interface for the YAMS server's Brainfuck\cite{Mpreu/preller}
interpreter. The page interacts with the interface through POST requests and
updates the text field on the page when it receives the result of running the
user-supplied Brainfuck code and input.

\section{Group Member Contributions}

\begin{LaTeXitemize} \itemsep0pt \parskip0pt
  \item Stephen Brennan
  \begin{enumerate}
    \item String functions
    \item Brainfuck interpreter
  \end{enumerate}

  \item Katherine Cass
  \begin{enumerate}
    \item Static site
    \item Report content
  \end{enumerate}

  \item Jeffrey Copeland
  \begin{enumerate}
    \item Socket syscalls
    \item HTTP request handling
  \end{enumerate}

  \item Andrew Mason
  \begin{enumerate}
    \item HTTP response building
    \item Project presentation
  \end{enumerate}

  \item Thomas Murphy
  \begin{enumerate}
    \item File access
    \item Code formatting and style
    \item Report organization and typesetting
  \end{enumerate}

\end{LaTeXitemize}

Aaron Neyer planned to build a website documenting our project, but could not
complete it due to personal circumstances.

\section{Conclusion}
We started this project to build a simple web server in MIPS. Despite the high
upfront workload, our collective components merged into what has been caled ``a
surprsingly robust webserver.'' Thus, we have achieved our goal: successfully
building a static webserver in MIPS and MARS.

\appendices

\onecolumn % use single column for the wide columns
\section{Screenshots of Code Execution Results}

\begin{figure}[H]
\centering
\includegraphics[width=0.8\textwidth,natwidth=1440,natheight=900]{static_document}
\caption{The static \texttt{index.html} served by YAMS by default.}
\label{fig:static_document}
\end{figure}

\begin{figure}[H]
\centering
\includegraphics[width=0.8\textwidth,natwidth=1440,natheight=900]{dynamic_document}
\caption{The first view of a dynamic HTML/JavaScript page containing the material presented to the class.}
\label{fig:dynamic_document}
\end{figure}

\begin{figure}[H]
\centering
\includegraphics[width=0.8\textwidth,natwidth=1440,natheight=900]{interactive_ajax}
\caption{A web interface to an implementation of the Brainfuck interpreter contained within YAMS.}
\label{fig:interactive_ajax}
\end{figure}

\twocolumn % return to two columns

% use section* for acknowledgment
\section*{Acknowledgment}


The authors would like to thank Cameron Gutman
\texttt{\url{https://github.com/cgutman}} for providing information about the
usage of identity transfer encoding.

% references section
\bibliography{IEEEabrv,./yams_report}
\bibliographystyle{IEEEtran}
\nocite{*}

% that's all folks
\end{document}
